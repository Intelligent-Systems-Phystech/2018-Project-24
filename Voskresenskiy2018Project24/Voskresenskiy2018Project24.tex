\documentclass[12pt,twoside]{article}

\usepackage{jmlda}
%\NOREVIEWERNOTES
\title
[Максимизация энтропии при различных видах преобразований над изображением] % Краткое название; не нужно, если полное название влезает в~колонтитул
{Максимизация энтропии при различных видах преобразований над изображением}
\author
[Автор~И.\,О.] % список авторов для колонтитула; не нужен, если основной список влезает в колонтитул
{Белозерцев~A.\,О., Воскресенский~Н.\,Д., Грибова~О.\,Б., Казаков~A.\,А., Мурзаев~Я.\,А., Хохлов~А.\,А., Шабалина~А.\,А.} % основной список авторов, выводимый в оглавление
[Белозерцев~A.\,О.$^1$, Воскресенский~Н.\,Д.$^1$, Грибова~О.\,Б.$^1$, Казаков~A.\,А.$^1$, Мурзаев~Я.\,А.$^1$, Хохлов~А.\,А.$^1$, Шабалина~А.\,А.$^1$] % список авторов, выводимый в заголовок; не нужен, если он не отличается от основного
\thanks
{Работа выполнена при финансовой поддержке РФФИ, проект \No\,00-00-00000. 
	Научный руководитель:  Стрижов~В.\,В. 
	Задачу поставил:  Матвеев~И.\,А.
	Консультант:  Консультант~И.\,О.}
\email
{author@site.ru}
\organization
{$^1$Московский физико-технический институт (государственный университет); $^2$Организация}
\abstract
{Данная работа посвящена исследованию вопроса повышения разрешения мультиспектральных изображений. Рассмотрены разные метрики оценки качества улучшения пространственного разрешения изображений, показана энтропия изображения как идентификатор потерь информации и ее корреляция с преобразованием над изображением. Предложены подход для анализа изображений, алгоритм повышения разрешения путем использования опорных изображений, метод оптимизации параметров данного алгоритма. Проведен сравнительный анализ с аналогичными подходами. Найдены условия максимизации энтропии восстановленного изображения.
	
	\bigskip
	\textbf{Ключевые слова}: \emph {ключевое слово, ключевое слово,
		еще ключевые слова}.}
\titleEng
{Entropy maximization in an image various types of transformations}
\authorEng
{Belozertsev~A.\,O.$^1$, Voskresenskiy~N.\,D.$^1$, Gribova~O.\,B.$^1$, Kazakov~A.\,A.$^1$, Murzayev~Y.\,A.$^1$, Khokhlov~A.\,A.$^1$, Shabalina~A.\,A.$^1$}
\organizationEng
{$^1$Moscow Institute of Physics and Technology (State University); $^2$Organization}
\abstractEng
{English abstract.
	
	\bigskip
	\textbf{Keywords}: \emph{keyword, keyword, more keywords}.}
\begin{document}
\maketitle
%\linenumbers
\section{Введение}
Основной целью данной работы является разработка алгоритма повышения пространственного разрешения мультиспектральных изображений и изображений с узким диапазоном частот. 

Предметом исследования являются изображения с различным набором частот, имеющие низкое пространственное разрешение, а также панхроматические и RGB-изображения.

В настоящее время аэрокосмическая съемка является основным инструментом для исследований в таких областях как георазведка, метеопрогнозирование, картография, экологический мониторинг и др. При работе со снимками поверхности земли наиболее острой является проблема низкого разрешения полученных трехканальных (RGB) и узкоспектральных изображений, влекущая за собой потерю информативности. Решению задачи повышения качества снимков и посвящена данная работа.

Потребность в получении снимков высокого качества возникает при анализе изображений для распознавания объектов \cite{visilter2009rus}, при мониторинге территорий на основе аэрокосмических данных в аграрной \cite{murynin2013} и нефтегазовой \cite{bondur2012aero} отраслях, для регистрации и прогнозирования морского волнения \cite{bondur2016rusvosstanovlenie,bondur2016rusoptimalniy}. Кроме того, повышение разрешения снимка может использоваться для  повышения точности навигации летательных аппаратов \cite{ishutin2016rus,visilter2016ruscomplexirovaniye}.

Для решения задачи улучшения качества снимков поверхности земли предлагается использовать методы машинного обучения, в частности нейронные сети. 

В работе \cite{gorokhovskiy2017rus} изложен вероятностный алгоритм повышения разрешения мультиспектрального изображения при помощи опорного снимка в виде панхроматического изображения более высокого качества. В \cite{gurchenkov2016rus,bochkareva2016rus} предложены методы улучшения качества, построенные на экстраполяции или объединении пространственных спектров.
Одним из основных преимуществ представленного решения является использование универсального метода - нейронной сети, который позволяет достичь высоких результатов. Однако, данный алгоритм имеет ряд недостатков, среди которых можно отметить отсутствие его физической интерпретации, а также необходимость наличия больших вычислительных мощностей для реализации.
Целью представленного эксперимента является создание модели нейронной сети, которая смогла бы увеличить пространственное разрешение лучше имеющихся на данный момент аналогов, основанных на аналитических подходах. В качестве данных использовались снимки с космических спутников.

\section{Заключение}
Здесь будет заключение

\bibliographystyle{unsrt} %unsrt - for sorted bibliography / plain - for unsertoed bibliography
\bibliography{Project24}

% Решение Программного Комитета:
%\ACCEPTNOTE
%\AMENDNOTE
%\REJECTNOTE
\end{document}
